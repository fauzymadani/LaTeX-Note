\documentclass{article}
\usepackage[a4paper, margin=1.5cm]{geometry}
\usepackage{amsmath, amsfonts, amssymb}
\usepackage{listings}
\usepackage{multicol}
\usepackage{titlesec}
\usepackage{xcolor}

\definecolor{codegreen}{rgb}{0,0.6,0}
\definecolor{codegray}{rgb}{0.5,0.5,0.5}
\definecolor{codepurple}{rgb}{0.58,0,0.82}
\definecolor{backcolour}{rgb}{0.95,0.95,0.92}

\lstdefinestyle{mystyle}{
    backgroundcolor=\color{backcolour},   
    commentstyle=\color{codegreen},
    keywordstyle=\color{magenta},
    numberstyle=\tiny\color{codegray},
    stringstyle=\color{codepurple},
    basicstyle=\ttfamily\footnotesize,
    breakatwhitespace=false,         
    breaklines=true,                 
    captionpos=b,                    
    keepspaces=true,                 
    numbers=left,                    
    numbersep=5pt,                  
    showspaces=false,                
    showstringspaces=false,
    showtabs=false,                  
    tabsize=2
}

\lstdefinestyle{plainstyle}{
  backgroundcolor={},   
  frame=none,           
  numbers=none,         
  basicstyle=\ttfamily, 
  breaklines=true,      
  showstringspaces=false, 
}

\lstset{style=mystyle}
\titleformat{\section}[hang]{\normalfont\Large\bfseries}{\thesection.}{1em}{}
\title{C Programming Exercise Solution for Beginner}

\begin{document}
  \maketitle

  \begin{abstract}
  i'm currently learning C and planning to improve my \LaTeX Skill. so i came up with this idea. This is only for my learning purpose. doing C problem while writing the solution and
  my approach/solution to a \LaTeX Note. i'm gonna explain my approach to solve given problem.
  \end{abstract}

  \section{Calculate product of two integers}
  
  Write a C program that accepts two integers from the user and calculates the product of the two integers.
  Test Data :
  
  \begin{lstlisting}[style=plainstyle]
  > Input the first integer: 25
  > Input the second integer: 15
  Expected Output:
  > Product of the above two integers = 375
  \end{lstlisting}
  
  \subsection{Solution}
  \begin{lstlisting}[language=C, caption=Defining required Variable]
    #include <stdio.h>

    #define HEIGHT 7
    #define WIDTH 5
    #define RADIUS 6
    #define M_PI 3.14159265358979323846
    
    // global scope variable declaration
    int num1, num2, result;
  \end{lstlisting}
  
  when you declaring a number using a \texttt{\#define}, It has no type. It is a simple text substitution. although PHI is already defined by C itself, but i found on a stack overflow thread, they were defining PHi with \texttt{\#define}.
  this kind of problem can be tackled easyly by creating a separate \textit{function} outside of \texttt{main()} function, and declaring those variable using \textit{a global scope variable} for efficiency and avoiding conflict within other function. 
  while defining a global scope variable, i'm creating a function to calculate the integers:
  \begin{lstlisting}[language=C, caption=function]
  void CalculateTwoIntegers(int num1, int num2){
    result = num1 + num2;
    printf("The result is: %d\n", result);
  }
  \end{lstlisting}

  \subsection{Pros And Cons}
  
  by storing a number in \texttt{\#define}, it's not the best practice in the modern C. because as i said, \textbf{It has no data types, hard to debug, there's no scope, and etc.} 
  i'd reccomend using a \texttt{const} variable instead. for example: \texttt{const double PI = 3.1415926535;}. a \texttt{\#define} is more suitable for storing a macro rather than a number for mathematical operation unless you know what are you doing.
  
  \section{Calculate average weight for purchases} 
  
  Write a C program that accepts two item's weight and number of purchases (floating point values) and calculates their average value.
  for example: 
  
  \begin{lstlisting}[style=plainstyle]
  Weight - Item1: 15
  No. of item1: 5
  Weight - Item2: 25
  No. of item2: 4
  Expected Output:
  Average Value = 19.444444
  \end{lstlisting}
  
  \subsection{Solution}
  
  by using math shit for this problem, i've figured out using average value for calculating the result, we can implement this solution to our code:
  
  \begin{equation}
  \text{Average Value} = \frac{(w_1 \times n_1) + (w_2 \times n_2)}{n_1 + n_2}
  \label{eq:average-value}
  \end{equation}

  \vspace{2pt}
  
  As shown in Equation~\ref{eq:average-value}, By creating a separate function and declaring the required variable first using \texttt{double} data types, finally we can implement the math formula above
  to our code solution.
  \begin{lstlisting}[language=C, caption=Code Solution]
  void CalculatePurchases(){
    double wi1, ci1, wi2, ci2, resultDouble;
    
    printf("Enter Weight for item 1: ");
    scanf("%lf", &wi1);

    printf("Enter Weight for ci1: ");
    scanf("%lf", &ci1);

    printf("Enter Weight for weight item 2: ");
    scanf("%lf", &wi2);

    printf("Enter Weight for ci2: ");
    scanf("%lf", &ci2);

    resultDouble = ((wi1 * ci1) + (wi2 * ci2)) / (ci1 + ci2);
    printf("The result is %lf\n", resultDouble);
  } 
  \end{lstlisting}
  
  \textbf{\textit{Why does \texttt{scanf()} require to use \& ?}} well, It needs to change the variable. Since all arguments in C are passed by value you need to pass a pointer if you want a function to be able to change a parameter.
  Here's a super-simple example showing it:
  
  \begin{lstlisting}[language=C, caption=scanf example]
    void nochange(int var) {
    // Here, var is a copy of the original number. &var != &value
    var = 1337;
    }
    void change(int *var) {
        // Here, var is a pointer to the original number. var == &value
        // Writing to `*var` modifies the variable the pointer points to
        *var = 1337;
    }
    int main() {
        int value = 42;
        nochange(value);
        change(&value);
        return 0;
    }
  \end{lstlisting}
  
  C function parameters are always "pass-by-value", which means that the function scanf only sees a copy of the current value of whatever you specify as the argument expression.
  In this case \&i is a pointer value that refers to the variable i. scanf can use this to modify i. If you passed i, then it would only see an uninitialized value, which (a) is UB, (b) is not sufficient information for scanf to know how to modify i.

  \section{Print employee ID and monthly salary}
  
  Write a C program that accepts an employee's ID, total worked hours in a month and the amount he received per hour. Print the ID and salary (with two decimal places) of the employee for a particular month.
  Test Data :
  
  \begin{lstlisting}[caption=Input, style=plainstyle]
    Input the Employees ID(Max. 10 chars): 0342
    Input the working hrs: 8
    Salary amount/hr: 15000
    > Expected Output:
    Employees ID = 0342
    Salary = U\$ 120000.00 
  \end{lstlisting}
  
  for this problem, we need thee type of data:
  \begin{enumerate}
    \item{ID, it can be \texttt{char, int, or an array}. depends on the problem.} 
    \item{working hours, with \texttt{int} data types.}
    \item{Salary amount, \texttt{float} or \texttt{double} is fine.}
  \end{enumerate}
  by using this math formula to calculate the result:
  
  \begin{equation}
  Salary = hoursworked \times rateperhour
  \label{eq:salary}
  \end{equation}

  \subsection{Solution}

  Thus, as shown in equation \ref{eq:salary}, we can implement this math formulae to our code. in my opinion, my approach is too common but i believe this is an efficient way. and more importantly, i don't care about your opinion.
  \begin{lstlisting}[language=C, caption=Code Implementation]
    int working_hours, id;
    double salary_amount_per_hour;

    float CalculateWorkingHours()
    {
      printf("Enter employees ID: ");
      scanf("%d", &id);
      
      printf("Enter working hours: ");
      scanf("%d", &working_hours);
    
      printf("Enter salary_amount per hours: ");
      scanf("%lf", &salary_amount_per_hour);
    
      return working_hours * salary_amount_per_hour;
    }
  \end{lstlisting}
  by creating a separate function, i'm using \texttt{float} to return a float data types when the function is called. if you want to return a more accurate data types
  i'd reccomend using \texttt{double} types. but in this case we're not performing a complex math operation so \texttt{float} is fine for me.
  Thus, i called the function and store it as a variable in order to print the result. 
  \begin{lstlisting}[language=C, caption=Call the function]
    int main()
    {
      float salary = CalculateWorkingHours();
      printf("Salary = U\$ %.2f\n", salary);
      return EXIT_SUCCESS;
    }
  \end{lstlisting}

\end{document}
